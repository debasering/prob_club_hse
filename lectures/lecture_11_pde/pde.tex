\documentclass{beamer}
\usepackage[utf8]{inputenc}     % Allows input encoding of Cyrillic characters
\usepackage[T2A]{fontenc}       % Cyrillic font encoding
\usepackage[russian,english]{babel}  % Language support for Russian and English

\usepackage{amsmath}
\usepackage{amsfonts}
\usepackage{amssymb}

\title{Partial Differential Equations from Probabilistic Perspective}
\subtitle{HSE FES Probability Theory Club}
\author{Sergei Tikhonov}
\date{June 2, 2024}

\begin{document}

\frame{\titlepage}

\tableofcontents
\newpage

\section{Big picture of PDE}

\begin{frame}

Big Picture of PDE


\end{frame}

\begin{frame}
\frametitle{Big picture of PDE}

Оператор Лапласа:

\[
\Delta := \frac{\partial^2}{\partial x_1^2} + ... + \frac{\partial^2}{\partial x_n^2} \quad \quad \quad n \geq 1
\]

Три важных линейных уравнения:

\begin{itemize}
    \item Laplace Equation:
\end{itemize}
\[
- \Delta u(x) = 0
\]

\begin{itemize}
    \item Heat Equation:
\end{itemize}
\[
\frac{\partial}{\partial t} u(t, x) - \Delta u(t, x) = 0
\]

\begin{itemize}
    \item Wave Equation:
\end{itemize}
\[
\frac{\partial^2}{\partial t^2} u(t, x) - \Delta u(t, x) = 0
\]

\end{frame}

\begin{frame}
\frametitle{Big picture of PDE}

PDE второго порядка:

\[
A u_{xx} + 2B u_{xy} + C u_{yy} + D u_{x} + E u_{y} + F = 0
\]

\begin{itemize}
    \item Elliptic PDE: $B^2 - AC < 0$
    \item Parabolic PDE: $B^2 - AC = 0$
    \item Hyperbolic PDE: $B^2 - AC > 0$
\end{itemize}

С прошлого слайда:

Laplace eq: \( A = 1 \), \( B = 0 \), \( C = 1 \), \( D = 0 \), \( E = 0 \), \( F = 0 \).

Heat eq: \( A = 1 \), \( B = 0 \), \( C = 0 \), \( D = 0 \), \( E = -1 \), \( F = 0 \)

Wave eq: \( A = 1 \), \( B = 0 \), \( C = -1 \), \( D = 0 \), \( E = 0 \), \( F = 0 \)

\;

Коэффициенты могут зависеть от $x$ и $t$, тогда вышеописанные условия должны быть выполнены для любых $x$ и $t$ из домейна


\end{frame}


\begin{frame}
\frametitle{Big picture of PDE}

\begin{itemize}
    \item "Weak"\ solutions
\end{itemize}

Иногда решение PDE или его производные не существуют pointwise (теория построенная на пространствах Соболева)

\begin{itemize}
    \item Euler Lagrange equations
\end{itemize}

Вместо решения PDE будем решать оптимизационную задачу (теория построенная на вариационном исчислении)

\end{frame}


\begin{frame}

Black Scholes Equation


\end{frame}

\begin{frame}
\frametitle{Уравнение Блэка Шоулза}
\begin{figure}
    \centering
    \includegraphics[width=0.95\textwidth]{BS.jpeg}
    \caption{Блэк, Шоулз, Мертон}
\end{figure}
\end{frame}


\section{Black Scholes Equation}
\begin{frame}
\frametitle{Уравнение Блэка Шоулза}

\[ \!\!\!\! \partial_t C(t, s) + r s \partial_s C(t, s) + \frac{1}{2} \sigma^2 s^2 \partial_{ss} C(t, s) = r C(t, s) \quad \quad s \in \mathbb{R}_+ \; t \in [0, T)\]

С терминальным условием \(F(T, s)\): 

\begin{itemize}
    \item для европейского опциона кол \(F(T, s) = \max(s(T) - K, 0)\);
    \item для европейского опциона пут \(F(T, s) = \max(K - s(T), 0)\);
\end{itemize}

Здесь:

\begin{itemize}
    \item $s(t)$ - цена базового актива (бездивидендная акция)
    \item $C(t, s)$ - цена опциона
    \item T - момент экспирации
    \item K - страйк 
    \item r - ставка дисконтирования
    \item $\sigma$ - волатильность
\end{itemize}

\end{frame}

\begin{frame}
\frametitle{Решение уравнения Блэка Шоулза для опциона колл}

Пусть $\Phi(\cdot)$ функция стандартного нормального распределения, тогда решением уравнения Блэка Шоулза с payoff колл опциона:

\[
C(t, s) = s \Phi(d_1) - K e^{-r(T - t)} \Phi(d_2)
\]
where

\[
d_1 = \frac{\log \bigg(\frac{s e^{r(T - t)}}{K}\bigg)}{\sigma \sqrt{T - t}} + \frac{\sigma \sqrt{T - t}}{2} 
\]
\[
d_2 = \frac{\log \bigg(\frac{s e^{r(T - t)}}{K}\bigg)}{\sigma \sqrt{T - t}} - \frac{\sigma \sqrt{T - t}}{2} 
\]

\end{frame}

\begin{frame}
\frametitle{Решение уравнения Блэка Шоулза}

Для "произвольного" терминального условия:

\[
C(t, s) = e^{-r(T-t)} \mathbb{E}^Q[F(T, S_T) \mid S_t = s]
\]

где \(Q\) - риск нейтральная мера.

\,

В самой модели мы предполгаем что цена акции имеет GBM динамику:

\[
d S_t = \mu S_t dt + \sigma S_t d W_t \quad \quad S_0 = s_0
\]

\end{frame}


\subsection{Probabilistic Representation of Black Scholes solution}
\begin{frame}
\frametitle{Формула Фейнмана Каца}
\begin{figure}
    \centering
    \includegraphics[width=0.45\textwidth]{Richard_Feynman_Nobel.jpg}
    \hfill
    \includegraphics[width=0.45\textwidth]{Mark_Kac.jpg}
    \caption{Слева: Ричард Фейнман. Справа: Марк Катц}
\end{figure}
\end{frame}

\begin{frame}
\frametitle{Обозначения}

Пусть \(S_t\) цена базового актива:

\[ dS_t = m(t, S_t) dt + \sigma(t, S_t) dW_t \quad S_0 = s_0 \]

Пусть \(R_t\) динамика процентных ставок:

\[d R_t = r(t, S_t) R_t dt\]

В "классической"\ модели Блэка Шоулза:

\[m(t, S_t) = \mu S_t \quad \quad \sigma(t, S_t) = \sigma S_t \quad \quad r(t, S_t) = r\]

Заметим, что динамика процентных ставок является ODE, и для классического случая имеем:

\[R_t = e^{rt}\]

\end{frame}

\begin{frame}
\frametitle{Идея формулы Фейнмана Каца}

Метод решения PDE с помощью аналитического вычисления математического ожидания / симуляций траекторий случайного процесса. Обратное тоже верно: важный класс математических ожиданий случайных процессов может быть вычислен как решения детерминированных моделей.

\end{frame}

\begin{frame}
\frametitle{Формула Фейнмана Каца}

Теорема (Фейнман-Кац формула). Пусть \(S_t\) удовлетворяет

\[ dS_t = m(t, S_t) dt + \sigma(t, S_t) dW_t \quad S_0 = s_0 \]

и \(r(t, s) \geq 0\) является ставкой дисконтирования. Пусть payoff \(F\) в момент времени \(T\) удовлетворяет \(\mathbb{E}[|F(S_T)|] < \infty\). Если \(C(t, s)\), \(0 \le t \le T\) определяется как:

\[ C(t, s) = \mathbb{E}\bigg[\frac{R_t}{R_T} F(S_T)\, \bigg|\, S_t = s\bigg]\]

причём \(C(t, s)\) является \(C^1\) по \(t\) и \(C^2\) по \(x\), то \(C(t, s)\) является решением PDE:

\[ \partial_t C(t, s) + m(t, s) \partial_s C(t, s) + \frac{1}{2} \sigma(t, s)^2 \partial_{ss} C(t, s) = r(t, s) C(t, s) \]

для \(0 \le t < T\), с терминальным условием \(C(T, s) = F(s)\).

\end{frame}

\begin{frame}
\frametitle{Доказательство формулы Фейнмана Каца}

Рассмотрим дисконтированный payoff $Y = R_T^{-1} F(S_T)$. Что можно сказать о $M_t = \mathbb{E}(Y \mid \mathcal{F}_t)$? $M_t$ является мартингалом (используем tower rule):

\[
\mathbb{E}(M_t \mid \mathcal{F}_s) = \mathbb{E}(\mathbb{E}(Y \mid \mathcal{F}_t)\mid \mathcal{F}_s) = \mathbb{E}(Y \mid \mathcal{F}_s) = M_s
\]

Более того, можем записать явную форму:

\[
\!\!\!\!\!\!\!\!\!\!\! M_t = \mathbb{E}(Y \mid \mathcal{F}_t) = \mathbb{E}(R_T^{-1} F(S_T) \mid \mathcal{F}_t) = 
\mathbb{E}\bigg[ \frac{1}{R_t} \frac{R_t}{R_T} F(S_T) \bigg| \mathcal{F}_t\bigg] = R_t^{-1} C(t, s)
\]

Хотим рассмотреть $M_t = R_t^{-1} C(t, s)$ через product rule и формулу Итô.

\end{frame}

\begin{frame}
\frametitle{Доказательство формулы Фейнмана Каца}

Product rule для случайных процессов (доказывается путём рассмотрения формулы Ито для $f(X_t, Y_t) = X_t Y_t$): 
\[
d(X_t Y_t) = X_t dY_t + Y_t d X_t + (d X_t) (d Y_t)
\]

Обратите внимание, что если $X_t$ и/или $Y_t$, представленные в виде процесса Ито, не содержат диффузии, то $(d X_t) (d Y_t)$ автоматически $0$.

В нашем случае:

\[
d(R_t^{-1}C(t, S_t)) = R_t^{-1} d C(t, S_t) + C(t, S_t) d R_t^{-1} + \underbrace{(d R_t^{-1})(d C(t, S_t))}_{=0}
\]

Осталось применить Ито для $C(t, S_t)$ и посчитать "обычный"\ дифференциал для $R_t^{-1}$.

\end{frame}

\begin{frame}
\frametitle{Доказательство формулы Фейнмана Каца}

Так как мы предположили что \(C(t, s)\) является \(C^1\) по \(t\) и \(C^2\), можем применить Ито для $C(t, S_t)$:

\[
d C(t, S_t) = \dot{C}(t, S_t)dt + C'(t, S_t) d S_t + \frac{1}{2}C''(t, S_t) (d S_t)^2 = 
\]
\[
= [\dot{C}(t, S_t) + m(t, S_t)C'(t, S_t) + \frac{1}{2} \sigma^2(t, S_t)C''(t, S_t)] dt +
\]
\[
+ C'(t, S_t) \sigma(t, S_t)d W_t
\]

И дифференциал $d[R_t^{-1}]$:
\[
\frac{d}{dt} \frac{1}{R_t} = - \frac{1}{R_t^2} d R_t = -\frac{1}{R_t}r(t, S_t) dt = - R_t^{-1} r(t, S_t) dt
\]

\end{frame}

\begin{frame}
\frametitle{Доказательство формулы Фейнмана Каца}

Подставим в product rule:

\[
d(R_t^{-1} C(t, S_t)) = R_t^{-1} [(\dot{C}(t, S_t) + m(t, S_t)C'(t, S_t) +
\]
\[
+ \frac{1}{2} \sigma^2(t, S_t)C''(t, X_t)) dt + C'(t, S_t) \sigma(t, S_t)d W_t] - R_t^{-1} r(t, S_t) C(t, S_t) dt 
\]

Выносим $R_t^{-1}$ за скобки и группируем слагаемые:

\[
d(R_t^{-1} C(t, S_t)) = [\dot{C}(t, S_t) +  m(t, S_t)C'(t, S_t) + \frac{1}{2} \sigma^2(t, S_t)C''(t, S_t) 
\]
\[
- r(t, S_t) C(t, S_t)] dt + C'(t, S_t) \sigma(t, S_t)d W_t
\]

Но мы показали, что \(R_t^{-1} C(t, S_t)\) - мартингал, а для мартингала дрифт равен $0$...

\end{frame}


\begin{frame}
\frametitle{Важная лемма}

Процесс Итô \(X_t\)

\[ dX_t = m(t, X_t) dt + \sigma(t, X_t) dW_t \]

является (локальным) мартингалом тогда и только тогда когда коэффициент при dt, дрифт, равен 0 почти наверное:

\[m(t, X_t) = 0 \quad \quad \quad \text{ a.s.}\]


Важно: каждый мартингал является локальным мартингалом, но не каждый локальный мартингал является мартингалом.

\end{frame}

\begin{frame}
\frametitle{Доказательство важной леммы}

Запишем процесс Итô \(X_t\) в интегральной форме: 

\[
X_t = X_0 + \int_0^t m(s, X_s) \, ds + \int_0^t \sigma(s, X_s) \, dW_s.
\]

Запишем определение мартингала:

\[
\mathbb{E}[X_t | \mathcal{F}_s] = \mathbb{E}\left[X_0 + \int_0^t m(u, X_u) \, du + \int_0^t \sigma(u, X_u) \, dW_u \bigg| \mathcal{F}_s\right] = 
\]

\[
= \underbrace{X_0 + \int_0^s m(u, X_u) \, du + \int_0^s \sigma(u, X_u) \, dW_u}_{X_s} +
\]

\[+ \mathbb{E}\bigg[\int_s^t m(u, X_u) \, du + \int_s^t \sigma(u, X_u) \, dW_u\bigg| \mathcal{F}_s\bigg] = 
\]

\end{frame}

\begin{frame}
\frametitle{Доказательство важной леммы}

Применим теорему Фубини для первого слагаемого, и определение интеграла Ито для второго слагаемого:

\[
= X_s + \mathbb{E}\bigg[ \int_s^t m(u, X_u) \, du \bigg| \mathcal{F}_s \bigg] + \underbrace{\mathbb{E}\bigg[\int_s^t \sigma(u, X_u) \, dW_u \bigg| \mathcal{F}_s \bigg]}_{0 \text{ по определению}} 
\]
\[
= X_s + \int_s^t \mathbb{E}[m(u, X_u) \mid \mathcal{F}_s] \, du = X_s
\]
Чтобы последнее равенство было выполнено, условие $m(t, X_t) = 0$ почти наверное для любого $t$ должно быть удовлетворено. 


\end{frame}

\begin{frame}
\frametitle{Проблема}

В результате получили:

\[ \partial_t C(t, s) + m(t, s) \partial_s C(t, s) + \frac{1}{2} \sigma(t, s)^2 \partial_{ss} C(t, s) = r(t, s) C(t, s) \]

В классической версии Блэка Шоулза:

\[
m(t, x) = \mu s \quad \quad \sigma(t, s) = \sigma s \quad \quad r(t, x) = r
\]

Полученное уравнение не является уравнением Блэка Шоулза:

\[ \partial_t C(t, s) + \mu s \partial_s C(t, s) + \frac{1}{2} \sigma^2 s^2 \partial_{ss} C(t, s) = r C(t, s) \]



\end{frame}

\begin{frame}
\frametitle{Теорема Гирсанова (чёрный ящик нашей лекции)}

Решение? Теорема Гирсанова... Гарантирует, что существует мера $Q$ такая что $W_t^Q$ будет являться Броуновским движением под мерой $Q$:

\[d W_t = A_t dt + d W_t^Q\]

Возьмём $A_t = \frac{r - \mu}{\sigma}$ и подставим $d W_t$ в:

\[ dS_t = \mu S_t dt + \sigma S_t dW_t \]

Получим:

\[ dS_t = r S_t dt + \sigma S_t dW_t^Q \]

Фейнман Кац применим, но теперь ожидание будет вычисляться под мерой $Q$:

\[ C(t, s) = \mathbb{E}^Q\bigg[\frac{R_t}{R_T} F(S_T)\, \bigg|\, S_t = s\bigg] = e^{-r(T-t)} \mathbb{E}^Q[F(S_T) \mid S_t = s] \]


\end{frame}

\subsection{Fourier Transform}

\begin{frame}

Fourier Transform


\end{frame}

\begin{frame}
\frametitle{Замены}

С помощью сложных и совершенно неинтуитивных замен мы можем показать, что изначальное уравнение Блэка Шоулза эквивалентно Heat equation (достаточно решить для \(V(\tau, x)\)):


\[
\frac{\partial V}{\partial \tau} = \frac{\partial^2 V}{\partial x^2}, \quad x \in \mathbb{R}, \ \tau \in [0, \sigma^2 T / 2],
\]

\[
S = K e^x,
\]

\[
\tau = (T - t) \sigma^2 / 2,
\]

\[
C(S, t) = K e^{-(a^2 / 4 + a + 1) \tau} e^{-(a / 2) x} V(x, \tau),
\]

\[
a = 2r / \sigma^2 - 1.
\]

Поскольку удалось избавиться от постоянных коэффициентов, можем применить преобразование Фурье и воспользоваться его свойствами. 
\end{frame}

\begin{frame}
\frametitle{Преобразование Фурье}

Определение:

\[ \hat{u}(\xi) = \mathcal{F} [u](\xi) = \int_\mathbb{R} e^{-ix \xi} u(x)dx\]

\[ u(x) = \mathcal{F}^{-1} [\hat{u}](x) = \frac{1}{2\pi}\int_\mathbb{R} e^{ix \xi} \hat{u}(\xi)d\xi\]

Полезные свойства для решения PDE:

\[ \mathcal{F}\{af(x) + bg(x)\} = a\hat{f}(\xi) + b\hat{g}(\xi) \]

\[ \mathcal{F}\left\{\frac{d^n f(x)}{dx^n}\right\} = (i\xi)^n \hat{f}(\xi) \]

\[ \mathcal{F}\{f(x) * g(x)\} = \hat{f}(\xi) \hat{g}(\xi) \]


\end{frame}


\begin{frame}
\frametitle{Решение PDE с помощью преобразования Фурье}

Применим преобразование фурье к $V(\tau, x)$ по переменной $x$:

\[
\mathcal{F}\bigg[\frac{\partial V}{\partial \tau} (\tau, x)\bigg] = \frac{\partial}{\partial \tau} \mathcal{F}\bigg[V(\tau, x)\bigg] = \frac{\partial}{\partial \tau} \hat{V}(\tau, \xi)
\]

\[
\mathcal{F}\bigg[\frac{\partial^2 V}{\partial x^2} (\tau, x)\bigg] = -\xi^2 \hat{V}(\tau, \xi)
\]

\[
\mathcal{F}[V(0, x)] = \hat{V}(0, \xi)
\]

Таким образом, получили ODE относительно $\tau$:

\[
\frac{\partial}{\partial \tau} \hat{V}(\tau, \xi) = -\xi^2 \hat{V}(\tau, \xi)
\]
ODE с разделяющимися переменными:
\[
\hat{V}(\tau, \xi) = \hat{V}(0, \xi) e^{-\xi^2 \tau}
\]

\end{frame}

\begin{frame}
\frametitle{Решение PDE с помощью преобразования Фурье}

Применим обратное преобразование Фурье:

\[
\mathcal{F}^{-1}[\hat{V}(0, \xi)] = V(0, x)
\]

\[
\Phi(x, t) = \mathcal{F}^{-1}[e^{-\xi^2 \tau}] = \frac{1}{\sqrt{4\pi \tau}} e^{-\frac{x^2}{4\tau}}
\]

Воспользуемся свойством о свёртке и запишем решение:

\[
V(\tau, x) = \int_\mathbb{R} \Phi(\tau, x-y)V(0, y) dy
\]

Заметим, что \(\Phi(x,\tau)\) является \(\Phi(x, \tau) > 0\) и \(\int_\mathbb{R} \Phi(x,\tau) dx = 1\). Значит, \(\Phi(x,\tau)\) является плотностью распределения. Более того, мы точно можем сказать, что \(X \sim N(0, 2\tau)\).


\end{frame}




\end{document}
