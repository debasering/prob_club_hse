\documentclass{beamer}% тип документа
\usepackage[utf8]{inputenc}
\usepackage[russian]{babel} 
\usepackage{graphicx}

\theoremstyle{definition}
\newtheorem{mydef}[theorem]{Определение}
\newtheorem{properties}[theorem]{Свойства}
\newtheorem{proposition}[theorem]{Утверждение}
\newtheorem{remark}[theorem]{Замечание}
\newtheorem{myexample}[theorem]{Пример}

\useoutertheme[footline=institutetitle]{miniframes}
% далее идёт преамбула
\title{Математическое ожидание и дисперсия}
\author[А. Макаров, Д. Тен]{А. Макаров, Д. Тен}
\institute[Клуб теории вероятностей]{Клуб теории вероятностей ФЭН ВШЭ}
\date{9 октября 2021}
\usetheme{Madrid}
\usepackage{graphics}

\newcommand\mytext{Основы Теории Вероятностей}

\usepackage{euscript}   % Шрифт Евклид
\usepackage{mathrsfs} % Красивый матшрифт

 \makeatother
 \setbeamercolor{footlinecolor}{bg=black!80,fg=white}
\setbeamertemplate{footline}
{%
  \leavevmode%
  \hbox{\begin{beamercolorbox}[wd=.5\paperwidth,ht=2.5ex,dp=1.125ex,leftskip=.3cm,rightskip=.3cm]{author in head/foot}%
  
  \mytext 
  \end{beamercolorbox}%

  \begin{beamercolorbox}[wd=.5\paperwidth,ht=2.5ex,dp=1.125ex,leftskip=.3cm,rightskip=.3cm plus1fil]{author in head/foot}%
    \usebeamerfont{author in head/foot}\insertshortauthor\hfill\insertpagenumber
  \end{beamercolorbox}}%
  \vskip0pt%
}
\makeatletter




\begin{document}  % начало презентации
\begin{frame}
\titlepage
\end{frame}


\begin{frame}{Мотивация}
\begin{itemize}
    \item Одни из основных объектов ТВ;
    \item Первое применение интеграла Лебега в нашем клубе;

\end{itemize}

\end{frame}


\begin{frame}{Введение матожидания}
    
    \begin{mydef}[Математическое ожидание]
    Говорится, что случайная величина f имеет конечное мат. ожидание, если $f \in \mathcal{L}(\Omega, \mathbb{P})$ . Тогда мы можем обозначить следующим образом 
    $\mathbb{E}f = \int_\Omega f \,d\mathbb{P}$ 
    \end{mydef}
    
\end{frame}

\begin{frame}{Введение дисперсии}
    
    \begin{mydef}[Дисперсия]
    Говорится, что случайная величина f имеет конечную дисперсию, если $f \in \mathcal{L}^2(\Omega, \mathbb{P})$ . Тогда 
    $var(f) = \int_\Omega (f - \mathbb{E}f)^2 \,d\mathbb{P} = \mathbb{E}(f - \mathbb{E}f)^2$
    \end{mydef}
    
    
\end{frame}

\begin{frame}{Замечание 1}
    
    А кто сказал, что дисперсия вообще существует? Может, такой интеграл равен бесконечности?
    
    
    Постоянная с.в. 1 является квадратично-интегрируемой по Лебегу, поэтому, по неравенству Коши-Буняковского-Шварца, если $f \in \mathcal{L}^2$, то f - интегрируема по Лебегу. Из этого следует интегрируемость такого выражения $(f - \mathbb{E}f)^2$
    
\end{frame}

\begin{frame}{Напоминание}
    
    Напомним, что можно задать меру на $\mathcal{B}(\mathbb{R})$ через функцию распределения: $\mu((-\infty,a]) = F_f(a) = Prob(f\leq a)$
    Заметим, что $\mu(A) = Prob(f \in A) = \mathbb{P}(f^{-1}(A))$
    
    Это значит, что мы теперь можем рассматривать $ \int_\mathbb{R}x\,d\mu = \int_\mathbb{R}x\,dF_f$
    
    
\end{frame}


\begin{frame}{Важная теорема}
\begin{proposition}
    Пусть функция $f$ имеет конечное матожидание. 
    
    Тогда $\mathbb{E}f = \int_\mathbb{R}x\,dF_f$
\end{proposition}

\end{frame}

\begin{frame}{Важнейшая теорема}



\begin{proposition}
    Пусть $f: \mathbb{R} \rightarrow \mathbb{R}$  - борелевская функция и $\xi$ - случайная величина. 
    
    Тогда $\mathbb{E}f(\xi)) = \int_\mathbb{R}f(x)\,dF_\xi$
\end{proposition}

\end{frame}

\begin{frame}{Следствие из важнейшей теоремы}

\begin{proposition}
    Для случайной величины $\xi$ выполнено следующее утверждение:
    
    $var(\xi) = \int_\mathbb{R}(x-\mathbb{E}(\xi))^2\,dF_\xi$
\end{proposition}
Доказательство: положим $ g(x) =(x-\mathbb{E}(\xi))^2 $

\end{frame}

\begin{frame}{Не менее важная теорема}

\begin{proposition}
    Пусть у нас есть случайная величина $\xi$ с функцией распределения вида $F_\xi(a) = \int_{(-\infty,a]}\phi(x)\,dx$ с какой-то неотрицательной борелевской функцией $\phi$, для которой выполнено $\int_\mathbb{R}\phi(x)\,dx =1$, где x - мера Лебега на $\mathbb{R}$.
    
    Тогда для борелевской фукнции $g:\mathbb{R} \rightarrow \mathbb{R}$ выполнено $\mathbb{E}g(\xi)) = \int_\mathbb{R}g(x)\phi(x)\,dx$
\end{proposition}
\end{frame}

\begin{frame}{Замечание 2} 
    Если $g(x)f(x)$ интегрируема по Риману, то $\int_\mathbb{R}g(x)\phi(x)\,dx$ принимает одно и то же значение и по Риману, и по Лебегу, поэтому можно применить формулу из школы(а потом забыть, это помойка истории) 
    
\end{frame}
\begin{frame}{Абсолютно непрерыная случайная величина}
    
    \begin{mydef}[Абсолютно непрерыная случайная величина]
    Говорится, что случайная величина $\xi$ является абсолютно непрерывной, если её функция распределения имеет вид: $F(\xi) = \int_-\infty^x\phi(t)\,dt$ с некоторой неотрицательной функцией $\phi$, заданной на $\mathbb{R}$, для которой выполнено $\int_\mathbb{R}\phi(t)\,dt =1$
    \end{mydef}
    Если $\phi$ абсолютно непрервна, тогда $F_\phi:\mathbb{R} \rightarrow [0, 1]$ непрерывна
    
    
\end{frame}



\end{document}