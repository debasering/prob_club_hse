\documentclass[a4paper, 12pt]{article}
\usepackage{comment} 
\usepackage{lipsum} 
\usepackage{fullpage} 
\usepackage[a4paper, total={7in, 10in}]{geometry}
\usepackage{setspace}
\onehalfspacing % полуторный интервал для всего текста

\usepackage[fleqn]{amsmath}
\usepackage{mathtools,amsmath}
\usepackage{amssymb,amsthm} % assumes amsmath package installed
\newtheorem{theorem}{Theorem}
\newtheorem{corollary}{Corollary}
\usepackage{graphicx}
\usepackage{tikz}
\usetikzlibrary{arrows}
\usepackage{verbatim}
\usepackage{float}
\usepackage{tikz}
\usetikzlibrary{automata,positioning}
\usepackage{pgfplots}
\usetikzlibrary{shapes,arrows}
\usetikzlibrary{arrows,calc,positioning}
\tikzset{
	block/.style = {draw, rectangle,
		minimum height=1cm,
		minimum width=1.5cm},
	input/.style = {coordinate,node distance=1cm},
	output/.style = {coordinate,node distance=4cm},
	arrow/.style={draw, -latex,node distance=2cm},
	pinstyle/.style = {pin edge={latex-, black,node distance=2cm}},
	sum/.style = {draw, circle, node distance=1cm},
}
\usepackage{xcolor}
\usepackage{mdframed}
\usepackage[shortlabels]{enumitem}
\usepackage{indentfirst}
\usepackage{hyperref}
\usepackage{wrapfig}

%% Работа с языком
\usepackage[utf8]{inputenc}
\usepackage[russian]{babel}
\usepackage{booktabs}
\usepackage{pgfplots}
\pgfplotsset{width=9cm, height=5cm, compat=1.9}

\renewcommand{\thesubsection}{\thesection.\alph{subsection}}

\newenvironment{problem}[2][Задача]
{ \begin{mdframed}[backgroundcolor=gray!20] \textbf{#1 #2} \\}
	{ \end{mdframed}}

% Define solution environment
\newenvironment{solution}
{\textit{Решение:}}
{}
\renewcommand{\qed}{\quad\qedsymbol}


\newcommand{\R}{\mathbb{R}}
\newcommand{\E}{\mathbb{E}}
\renewcommand{\P}{\mathbb{P}}
\renewcommand{\F}{\mathcal{F}}
\DeclareMathOperator{\Var}{\mathbb{V}ar}
\DeclareMathOperator{\Cov}{\mathbb{C}ov}
\DeclareMathOperator{\Corr}{\mathbb{C}orr}
\DeclareMathOperator{\sign}{sign}
\DeclareMathOperator{\rank}{rank}
\DeclareMathOperator{\plim}{plim}

\title{Задачи для Клуба теории вероятностей ФЭН ВШЭ}
\author{Никита Киселев}
\date{9 июня 2022}

\begin{document}

\maketitle

\section*{Задача 1}
Prove \textbf{Chebyshev’s inequality} - if $a>0$ then
$$\P(|X|\geqslant a|\F)\leqslant a^{-2}\mathbb{E}(X^2|\F)$$


\section*{Задача 2}
Show that if $X$ and $Y$ are random variables with $\mathbb{E}(Y|\F)=X$ and $\mathbb{E}Y^2=\mathbb{E}X^2<\infty$, then $X=Y \: a.s.$


\section*{Задача 3\dagger}
The result in the last exercise implies that if $\mathbb{E}Y^2<\infty$ and
$\mathbb{E}(Y|\F)$ has the same distribution as $Y$, then $\mathbb{E}(Y|\F)=Y \: a.s.$ Prove this under the assumption $\mathbb{E}Y<\infty$.


\section*{Задача 4}
A coin shows heads with probability $p$. Let $X_n$ be the number of flips required to obtain a run of $n$ consecutive heads. Show that
$$\mathbb{E}X_n=\sum\limits_{k=1}^np^{-k}$$

\section*{Задача 5}
An urn contains initially $b$ blue balls and $r$ red balls, where $b$, $r\geqslant2$. Balls are drawn one by one
without replacement. Show that the mean number of draws until the first colour drawn is first repeated
equals $3$.


\end{document}
