\documentclass[a4paper, 12pt]{article}
\usepackage{comment} 
\usepackage{lipsum} 
\usepackage{fullpage} 
\usepackage[a4paper, total={7in, 10in}]{geometry}
\usepackage{setspace}
\onehalfspacing % полуторный интервал для всего текста

\usepackage[fleqn]{amsmath}
\usepackage{mathtools,amsmath}
\usepackage{amssymb,amsthm} % assumes amsmath package installed
\newtheorem{theorem}{Theorem}
\newtheorem{corollary}{Corollary}
\usepackage{graphicx}
\usepackage{tikz}
\usetikzlibrary{arrows}
\usepackage{verbatim}
\usepackage{float}
\usepackage{tikz}
\usetikzlibrary{automata,positioning}
\usepackage{pgfplots}
\usetikzlibrary{shapes,arrows}
\usetikzlibrary{arrows,calc,positioning}
\tikzset{
	block/.style = {draw, rectangle,
		minimum height=1cm,
		minimum width=1.5cm},
	input/.style = {coordinate,node distance=1cm},
	output/.style = {coordinate,node distance=4cm},
	arrow/.style={draw, -latex,node distance=2cm},
	pinstyle/.style = {pin edge={latex-, black,node distance=2cm}},
	sum/.style = {draw, circle, node distance=1cm},
}
\usepackage{xcolor}
\usepackage{mdframed}
\usepackage[shortlabels]{enumitem}
\usepackage{indentfirst}
\usepackage{hyperref}
\usepackage{wrapfig}

%% Работа с языком
\usepackage[utf8]{inputenc}
\usepackage[russian]{babel}
\usepackage{booktabs}
\usepackage{pgfplots}
\pgfplotsset{width=9cm, height=5cm, compat=1.9}

\renewcommand{\thesubsection}{\thesection.\alph{subsection}}

\newenvironment{problem}[2][Задача]
{ \begin{mdframed}[backgroundcolor=gray!20] \textbf{#1 #2} \\}
	{ \end{mdframed}}

% Define solution environment
\newenvironment{solution}
{\textit{Решение:}}
{}
\renewcommand{\qed}{\quad\qedsymbol}


\newcommand{\R}{\mathbb{R}}
\newcommand{\E}{\mathbb{E}}
\renewcommand{\P}{\mathbb{P}}
\DeclareMathOperator{\Var}{\mathbb{V}ar}
\DeclareMathOperator{\Cov}{\mathbb{C}ov}
\DeclareMathOperator{\Corr}{\mathbb{C}orr}
\DeclareMathOperator{\sign}{sign}
\DeclareMathOperator{\rank}{rank}
\DeclareMathOperator{\plim}{plim}

\title{Задачи для Клуба теории вероятностей ФЭН ВШЭ}
\author{Николай Аверьянов}
\date{9 сентября 2021}

\begin{document}

\maketitle

\section*{Задача 1}
Доказать, что если $F(x)$ -- функция распределения, то при любом $h \neq 0$ функции

$$
\Phi(x) = \frac{1}{h} \int \limits_{x}^{x+h} F(t) dt, \;\;\;\; \Psi(x) = \frac{1}{2h} \int \limits_{x-h}^{x+h} F(t)dt
$$

также являются функциями распределения.

\section*{Задача 2}
Пусть $\alpha(x) = x + [x]$, где квадратные скобки означают взятие целой части числа. Вычислите следуюший интеграл:

$$
\int \limits_0^{10} x d\alpha(x)
$$

\section*{Задача 3} 
Функции распределения $F_1(x)$ и $F_2(x)$ удовлетворяют условию 

$$
F_1(x) \leq F_2(x) \; \forall x.
$$

Показать, что можно так задать на одном вероятностном пространстве случайные величины $\xi_1$ и $\xi_2$ с функциями распределения $F_1(x)$ и $F_2(x)$ соответственно, что

$$
\P \{\xi_1 \geq \xi_2\} = 1
$$

\section*{Задача 4}
Покажите, что функция распределения $F_{\xi}(x)$ непрерывна в точке $x=x_0$ тогда и только тогда, когда $\P\{\xi = x_0\} = 0$


\section*{Задача 5}
Доказать, что любая функция распределения обладает следующими свойствами:

$$
\lim_{x \to \infty} x \int \limits_{x}^{\infty} \frac{1}{z} dF(z) = 0, \;\;\;\; \lim_{x \to +0} x  \int \limits_{x}^{\infty} \frac{1}{z} dF(z) = 0, 
$$

$$
\lim_{x \to -\infty} x \int \limits_{-\infty}^{x} \frac{1}{z} dF(z) = 0, \;\;\;\; \lim_{x \to -0} x  \int \limits_{-\infty}^{x} \frac{1}{z} dF(z) = 0, 
$$

\end{document}
