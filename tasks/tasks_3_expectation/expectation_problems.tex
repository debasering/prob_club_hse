\documentclass[a4paper, 12pt]{article}
\usepackage{comment} 
\usepackage{lipsum} 
\usepackage{fullpage} 
\usepackage[a4paper, total={7in, 10in}]{geometry}
\usepackage{setspace}
\onehalfspacing % полуторный интервал для всего текста

\usepackage[fleqn]{amsmath}
\usepackage{mathtools,amsmath}
\usepackage{amssymb,amsthm} % assumes amsmath package installed
\newtheorem{theorem}{Theorem}
\newtheorem{corollary}{Corollary}
\usepackage{graphicx}
\usepackage{tikz}
\usetikzlibrary{arrows}
\usepackage{verbatim}
\usepackage{float}
\usepackage{tikz}
\usetikzlibrary{automata,positioning}
\usepackage{pgfplots}
\usetikzlibrary{shapes,arrows}
\usetikzlibrary{arrows,calc,positioning}
\tikzset{
	block/.style = {draw, rectangle,
		minimum height=1cm,
		minimum width=1.5cm},
	input/.style = {coordinate,node distance=1cm},
	output/.style = {coordinate,node distance=4cm},
	arrow/.style={draw, -latex,node distance=2cm},
	pinstyle/.style = {pin edge={latex-, black,node distance=2cm}},
	sum/.style = {draw, circle, node distance=1cm},
}
\usepackage{xcolor}
\usepackage{mdframed}
\usepackage[shortlabels]{enumitem}
\usepackage{indentfirst}
\usepackage{hyperref}
\usepackage{wrapfig}

%% Работа с языком
\usepackage[utf8]{inputenc}
\usepackage[russian]{babel}
\usepackage{booktabs}
\usepackage{pgfplots}
\pgfplotsset{width=9cm, height=5cm, compat=1.9}

\renewcommand{\thesubsection}{\thesection.\alph{subsection}}

\newenvironment{problem}[2][Задача]
{ \begin{mdframed}[backgroundcolor=gray!20] \textbf{#1 #2} \\}
	{ \end{mdframed}}

% Define solution environment
\newenvironment{solution}
{\textit{Решение:}}
{}
\renewcommand{\qed}{\quad\qedsymbol}


\newcommand{\R}{\mathbb{R}}
\newcommand{\E}{\mathbb{E}}
\renewcommand{\P}{\mathbb{P}}
\DeclareMathOperator{\Var}{\mathbb{V}ar}
\DeclareMathOperator{\Cov}{\mathbb{C}ov}
\DeclareMathOperator{\Corr}{\mathbb{C}orr}
\DeclareMathOperator{\sign}{sign}
\DeclareMathOperator{\rank}{rank}
\DeclareMathOperator{\plim}{plim}

\title{Задачи для Клуба теории вероятностей ФЭН ВШЭ}
\author{Александр Плахин, Николай Аверьянов}
\date{9 октября 2021}

\begin{document}

\maketitle

\section*{Задача 1}
$X \sim \mathcal{N}(\mu, \sigma^2)$, $g$ -- некоторая дифференцируемая функция. Докажите следующее соотношение (если обе части определены):
$$
\E[(X-\mu)g(X)] = \sigma^2\E[g'(X)]
$$


\section*{Задача 2}

$X$ -- случайная величина с математическим ожиданием $\mu$ и функцией распределения $F$.
\begin{enumerate}
    \item Покажите, что
    $$
    \int_0^{\infty} (1-F(x))dx - \int_{-\infty}^0 F(x)dx = \mu
    $$
\item Покажите, что 
$$
\int_{-\infty}^a F(x)dx = \int_a^{\infty} (1-F(x))dx \iff a = \mu.
$$

\end{enumerate}

\section*{Задача 3}
На бесконечный лист клетчатой бумаги (сторона клеточки равна $1$) случайно бросается круг единичного радиуса. Считая, что центр круга равномерно распределен на том единичном квадрате, на который он попал, найти математическое ожидание числа точек с целочисленными координатами $(x, \; y)$, покрытых этим кругом.


\section*{Задача 4}

\begin{enumerate}
    \item Пусть $X$ -- случайная величина такая, что $\E[X] = 0$, $\Var[X] = \sigma^2 < \infty$. Докажите, что 
    $$
    \P\{X \geq a\} \leq \cfrac{\sigma^2}{\sigma^2 + a^2}, \; a > 0
    $$
    \item Теперь предположим, что $X$ -- нестрого положительная случайная величина с конечным вторым моментом. Докажите, что
    $$
    \P\{X > a\} \geq \cfrac{(\E[X])^2}{\E[X^2]}
    $$
\end{enumerate}


\end{document}