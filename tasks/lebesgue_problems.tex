\documentclass[a4paper, 12pt]{article}
\usepackage{comment} 
\usepackage{lipsum} 
\usepackage{fullpage} 
\usepackage[a4paper, total={7in, 10in}]{geometry}
\usepackage{setspace}
\onehalfspacing % полуторный интервал для всего текста

\usepackage[fleqn]{amsmath}
\usepackage{mathtools,amsmath}
\usepackage{amssymb,amsthm} % assumes amsmath package installed
\newtheorem{theorem}{Theorem}
\newtheorem{corollary}{Corollary}
\usepackage{graphicx}
\usepackage{tikz}
\usetikzlibrary{arrows}
\usepackage{verbatim}
\usepackage{float}
\usepackage{tikz}
\usetikzlibrary{automata,positioning}
\usepackage{pgfplots}
\usetikzlibrary{shapes,arrows}
\usetikzlibrary{arrows,calc,positioning}
\tikzset{
	block/.style = {draw, rectangle,
		minimum height=1cm,
		minimum width=1.5cm},
	input/.style = {coordinate,node distance=1cm},
	output/.style = {coordinate,node distance=4cm},
	arrow/.style={draw, -latex,node distance=2cm},
	pinstyle/.style = {pin edge={latex-, black,node distance=2cm}},
	sum/.style = {draw, circle, node distance=1cm},
}
\usepackage{xcolor}
\usepackage{mdframed}
\usepackage[shortlabels]{enumitem}
\usepackage{indentfirst}
\usepackage{hyperref}
\usepackage{wrapfig}

%% Работа с языком
\usepackage[utf8]{inputenc}
\usepackage[russian]{babel}
\usepackage{booktabs}
\usepackage{pgfplots}
\pgfplotsset{width=9cm, height=5cm, compat=1.9}

\renewcommand{\thesubsection}{\thesection.\alph{subsection}}

\newenvironment{problem}[2][Задача]
{ \begin{mdframed}[backgroundcolor=gray!20] \textbf{#1 #2} \\}
	{ \end{mdframed}}

% Define solution environment
\newenvironment{solution}
{\textit{Решение:}}
{}
\renewcommand{\qed}{\quad\qedsymbol}


\newcommand{\R}{\mathbb{R}}
\newcommand{\E}{\mathbb{E}}
\renewcommand{\P}{\mathbb{P}}
\DeclareMathOperator{\Var}{\mathbb{V}ar}
\DeclareMathOperator{\Cov}{\mathbb{C}ov}
\DeclareMathOperator{\Corr}{\mathbb{C}orr}
\DeclareMathOperator{\sign}{sign}
\DeclareMathOperator{\rank}{rank}
\DeclareMathOperator{\plim}{plim}

\title{Задачи для Клуба теории вероятностей ФЭН ВШЭ}
\author{Александр Плахин, Николай Аверьянов}
\date{25 сентября 2021}

\begin{document}

\maketitle

\section*{Задача 1}
Докажите следующее утверждение:
$$
\int_{[0, 1]}f d\mu = \inf_{\varphi \geq f} \int_{[0, 1]} f d\mu, 
$$
где инфимум берется по простым функция $\varphi$. На какое утверждение связанное с интегралом Римана похож данный факт? 

\section*{Задача 2}
Будем говорить, что $f \sim g$ относительно меры $\mu$, если $\mu(\{x: f(x) = g(x)\}) = 0$. 
\begin{enumerate}
    \item Покажите, что $\sim$ -- отношение эквивалентности.
    \item Докажите, что если $f \sim g$, то 
    $$
    \int_E f d\mu = \int_E g d\mu, \; \forall E \in \Sigma
    $$
\end{enumerate}

\section*{Задача 3} 

Докажите, что функция Дирихле $D(x)$ не интегрируема по Риману, но интегрируема относительно меры Лебега. Напомним, что 
$$
D(x) = \begin{cases}
1, x \in \mathbb{Q} \\
0, x \in \R \textbackslash \mathbb{Q}
\end{cases}
$$

\section*{Задача 4}
Вычислите следующий предел, воспользовавшись теоремой Лебега о мажорируемой сходимости:
$$
\lim_{n \to \infty} \int_{0}^{\infty} \cfrac{\ln(nx)}{x+x^2\ln n}dx 
$$

\end{document}