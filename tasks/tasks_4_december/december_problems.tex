\documentclass[a4paper, 12pt]{article}
\usepackage{comment} 
\usepackage{lipsum} 
\usepackage{fullpage} 
\usepackage[a4paper, total={7in, 10in}]{geometry}
\usepackage{setspace}
\onehalfspacing % полуторный интервал для всего текста

\usepackage[fleqn]{amsmath}
\usepackage{mathtools,amsmath}
\usepackage{amssymb,amsthm} % assumes amsmath package installed
\newtheorem{theorem}{Theorem}
\newtheorem{corollary}{Corollary}
\usepackage{graphicx}
\usepackage{tikz}
\usetikzlibrary{arrows}
\usepackage{verbatim}
\usepackage{float}
\usepackage{tikz}
\usetikzlibrary{automata,positioning}
\usepackage{pgfplots}
\usetikzlibrary{shapes,arrows}
\usetikzlibrary{arrows,calc,positioning}
\tikzset{
	block/.style = {draw, rectangle,
		minimum height=1cm,
		minimum width=1.5cm},
	input/.style = {coordinate,node distance=1cm},
	output/.style = {coordinate,node distance=4cm},
	arrow/.style={draw, -latex,node distance=2cm},
	pinstyle/.style = {pin edge={latex-, black,node distance=2cm}},
	sum/.style = {draw, circle, node distance=1cm},
}
\usepackage{xcolor}
\usepackage{mdframed}
\usepackage[shortlabels]{enumitem}
\usepackage{indentfirst}
\usepackage{hyperref}
\usepackage{wrapfig}

%% Работа с языком
\usepackage[utf8]{inputenc}
\usepackage[russian]{babel}
\usepackage{booktabs}
\usepackage{pgfplots}
\pgfplotsset{width=9cm, height=5cm, compat=1.9}

\renewcommand{\thesubsection}{\thesection.\alph{subsection}}

\newenvironment{problem}[2][Задача]
{ \begin{mdframed}[backgroundcolor=gray!20] \textbf{#1 #2} \\}
	{ \end{mdframed}}

% Define solution environment
\newenvironment{solution}
{\textit{Решение:}}
{}
\renewcommand{\qed}{\quad\qedsymbol}


\newcommand{\R}{\mathbb{R}}
\newcommand{\E}{\mathbb{E}}
\renewcommand{\P}{\mathbb{P}}
\DeclareMathOperator{\Var}{\mathbb{V}ar}
\DeclareMathOperator{\Cov}{\mathbb{C}ov}
\DeclareMathOperator{\Corr}{\mathbb{C}orr}
\DeclareMathOperator{\sign}{sign}
\DeclareMathOperator{\rank}{rank}
\DeclareMathOperator{\plim}{plim}

\title{Задачи для Клуба теории вероятностей ФЭН ВШЭ}
\author{Александр Плахин, Николай Аверьянов}
\date{4 декабря 2021}

\begin{document}

\maketitle

\section*{Задача 1}

\begin{itemize}
    \item Пусть $X$ и $Y$ имееют совместную плотность $f$. Найдите плотность $Y/X$.
    \item Пусть теперь $X$ и $Y$ независимы и одинаково распределены с плотностью $f$. Покажите, то $\arctan(Y/X)$ имеет равномерное распределение на $(-\frac{1}{2}\pi, \frac{1}{2}\pi)$ тогда и только тогда, когда:
    $$
    \int_{-\infty}^{\infty} f(x)f(xy)|x|dx = \cfrac{1}{\pi(1+y^2)}
    $$
\end{itemize}

\section*{Задача 2}
Четыре точки выбираются равновероятно внутри треугольника. Найти вероятность того, что ни одна точка не лежит внутри треугольника, образованного тремя остальными точками.

\section*{Задача 3}
Последовательность точек $\xi_1, \xi_2, \ldots$ на отрезке $[0, 1]$ строится по следующему правилу: $\xi_1 \sim U[0, 1]$, и если значения $\xi_1, \ldots, \xi_{k-1} \; (k \geq 2)$  определены, то точка $\xi_k$ имеет равномерное распределение на минимальном по длине из $k$ отрезков, на которые $[0, 1]$ разбивается точками $\xi_1, \ldots, \xi_{k-1}$.

\begin{itemize}
    \item Доказать, что существует случайная величина $\xi$, удовлетворяющая условию: 
    $$
    \P(\{\lim_{n \to \infty} \xi_n = \xi \}) = 1
    $$
    \item Найти $\E \xi, \; \Var \xi.$
\end{itemize}


\section*{Задача 4}
Предположим, что абсолютно непрерывное распределение вероятности скалярной случайной величины $X$ обладает следующим свойством: плотность совместного распределения набора $n$ независимых случайных величин, распределенных по такому же закону, зависит только от радиальной координаты $\sqrt{\sum_i x_i^2}$. Показать, что распределение $X$ нормально (точнее, гауссово с нулевым математическим ожиданием).


\end{document}