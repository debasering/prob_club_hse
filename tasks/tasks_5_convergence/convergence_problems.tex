\documentclass[a4paper, 12pt]{article}
\usepackage{comment} 
\usepackage{lipsum} 
\usepackage{fullpage} 
\usepackage[a4paper, total={7in, 10in}]{geometry}
\usepackage{setspace}
\onehalfspacing % полуторный интервал для всего текста

\usepackage[fleqn]{amsmath}
\usepackage{mathtools,amsmath}
\usepackage{amssymb,amsthm} % assumes amsmath package installed
\newtheorem{theorem}{Theorem}
\newtheorem{corollary}{Corollary}
\usepackage{graphicx}
\usepackage{tikz}
\usetikzlibrary{arrows}
\usepackage{verbatim}
\usepackage{float}
\usepackage{tikz}
\usetikzlibrary{automata,positioning}
\usepackage{pgfplots}
\usetikzlibrary{shapes,arrows}
\usetikzlibrary{arrows,calc,positioning}
\tikzset{
	block/.style = {draw, rectangle,
		minimum height=1cm,
		minimum width=1.5cm},
	input/.style = {coordinate,node distance=1cm},
	output/.style = {coordinate,node distance=4cm},
	arrow/.style={draw, -latex,node distance=2cm},
	pinstyle/.style = {pin edge={latex-, black,node distance=2cm}},
	sum/.style = {draw, circle, node distance=1cm},
}
\usepackage{xcolor}
\usepackage{mdframed}
\usepackage[shortlabels]{enumitem}
\usepackage{indentfirst}
\usepackage{hyperref}
\usepackage{wrapfig}

%% Работа с языком
\usepackage[utf8]{inputenc}
\usepackage[russian]{babel}
\usepackage{booktabs}
\usepackage{pgfplots}
\pgfplotsset{width=9cm, height=5cm, compat=1.9}

\renewcommand{\thesubsection}{\thesection.\alph{subsection}}

\newenvironment{problem}[2][Задача]
{ \begin{mdframed}[backgroundcolor=gray!20] \textbf{#1 #2} \\}
	{ \end{mdframed}}

% Define solution environment
\newenvironment{solution}
{\textit{Решение:}}
{}
\renewcommand{\qed}{\quad\qedsymbol}


\newcommand{\R}{\mathbb{R}}
\newcommand{\E}{\mathbb{E}}
\renewcommand{\P}{\mathbb{P}}
\DeclareMathOperator{\Var}{\mathbb{V}ar}
\DeclareMathOperator{\Cov}{\mathbb{C}ov}
\DeclareMathOperator{\Corr}{\mathbb{C}orr}
\DeclareMathOperator{\sign}{sign}
\DeclareMathOperator{\rank}{rank}
\DeclareMathOperator{\plim}{plim}

\title{Задачи для Клуба теории вероятностей ФЭН ВШЭ}
\author{Саша Плахин, Коля Аверьянов}
\date{18 февраля 2021}

\begin{document}

\maketitle


\section*{Задача 1}
\begin{itemize}
    \item Доказать, что если $X \geq 0$ и $\E X = 0$, то $X = 0$ почти наверное.
    \item Доказать, что если $\Var X = 0$, то $X$ почти наверное константа.
\end{itemize}


\section*{Задача 2}
Пусть $\{X_n: n \geq 1\}$ -- последовательность независимых экспоненциальных случайных величин с параметром $\lambda = 1$. Докажите:
$$
\P\bigg(\bigg\{\limsup_{n \to \infty} \cfrac{X_n}{\log n} = 1  \bigg\}\bigg) = 1.
$$

\section*{Задача 3}

Пусть $\{X_r, r \geq 1\}$
независимые пуассоновские случайные величины с параметрами $\{\lambda_r, r \geq 1\}$. Нужно показать, что $\sum_{r=1}^{\infty}X_r$сходится или расходится почти наверное в соответсвии с тем сходится или расходится
 $\sum_{r=1}^{\infty} \lambda_r$.

\section*{Задача 4}
\begin{itemize}
    \item Докажите, что из сходимости в среднем порядка $r$ следует сходимость в среднем порядка $r - k$, $k \in \{1, \ldots, r-1\}$
    \item Докажите, что из сходимости в среднем следует сходимость по вероятности
    \item Докажите, что $X_n \to 0$ по вероятности тогда и только тогда, когда:
    $$
    \lim_{n \to \infty} \E \bigg( \cfrac{|X_n|}{1+|X_n|}\bigg) = 0.
    $$
\end{itemize}


\section*{Задача 5}
Пусть $\xi_n \to \xi$ по распределению. $h(x)$ -- дважды дифференцируемая функция в вещественной точке $a$. Известно, что $h'(a) = 0$. Найдите предел сходимости по распределению у последовательности:
$$
\cfrac{h(a + \xi_n b_n) - h(a)}{b_n^2},
$$
где $b_n \to 0$ -- произвольная последовательность положительных чисел.

\end{document}